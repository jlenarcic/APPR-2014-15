\documentclass[11pt,a4paper]{article}

\usepackage[slovene]{babel}
\usepackage[utf8x]{inputenc}
\usepackage{graphicx}
\usepackage{pdfpages}
\usepackage{hyperref}
\usepackage{breakurl}

\pagestyle{plain}

\begin{document}
\title{Poročilo pri predmetu \\
Analiza podatkov s programom R}
\author{Jakob Lenarčič}
\maketitle

\section{Izbira teme}

V mojem projektu se bom ukvarjal s podrobno analizo NBA ekip v sezoni 2013/2014. Za vsako ekipo me bo zanimalo število zmag in porazov, povprečno število košev na tekmo in druge spremenljivke. Nekatere bom dodal sam (npr. ali je igralec v moštvu dober strelec; kriterij za dobrega strelca pa bo recimo več kot 20 točk na tekmo). Primerjal bom dve statistični spremenljivki, in sicer povprečno število košev na tekmo in število zmag. Podatke za vsako ekipo bom primerjal s podatki za ostale ekipe in na podlagi tega sklepal, ali je ekipa svoj uspeh gradila na čvrsti obrambi ali na zaneslivem napadu. Na podlagi vsake statistične spremenljivke bom ekipe razvrstil od najboljše do najslabše, nato pa bom ugotavljal, katere pozitivno vplivajo na njihovo uspešnost.
Prav tako bom uvozil še eno podobno tabelo. Razlikovala pa se bo v tem, da bodo na tej le podatki za tekme v gosteh. Ti dve tabeli bom med seboj primerjal, saj me bo zanimalo, v kakšni meri vpliva kraj tekme (doma, v gosteh) na uspešnost ekipe.
V drugem delu projekta pa se bom osredotočil na NBA ekipo Phoenix Suns. Zanimali me bosta predvsem dve spremenljivki, in sicer strelska uspešnost in povprečnen čas, ki ga igralec preživi na parketu. Ti dve   spremenljivki mi bosta v pomoč pri ugotavljanju, kateri izmed igralcev je bil najučinkovitejši glede na minutažo.



\section{Obdelava, uvoz in čiščenje podatkov}

Podatke, ki sem jih pridobil iz spletnih strani, navedenih v prvi fazi, sem prekopiral v Excel in jih shranil kot CSV datoteke. Potem sem te datoteke uvozil v R STUDIO. Dvema od treh tabel sem še dodal po eno statistično spremenljivko (urejenostno), in sicer tabelama z naslovom NBA1 in phoenix. Podatke sem nato predstavil še grafično. Sledijo predstavitve tabel in grafov za drugo fazo.
\newpage

Tabela \textbf{NBASTATES} (v CSV obliki) vsebuje 29 stolpcev. Za vsako NBA ekipo so podani naslednji podatki:

\begin{enumerate}
\item{\verb|ime NBA ekipe| (imenska spremenljivka),}
\item{\verb|odigrane tekme| (številska spremenljivka),}
\item{\verb|zmage| (številska spremenljivka),}
\item{ \verb|porazi| (številska),}
\item{\verb|odstotek zmag| (številska),}
\item{\verb|povprečno število minut, ki jih je ekipa odigrala na tekmo (tudi podaljški)| (številska),}
\item{\verb|povprečno število zadetih metov| (številska),}
\item{\verb|povprečno število vseh metov| (številska) ,}
\item{\verb|odstotek zadetih metov| (številska),}
\item{\verb|povprečno število zadetih metov za tri| (številska spremenljivka),}
\item{\verb|povprečno število vseh metov za tri| (številska),}
\item{ \verb|odstotek zadetih trojk| (številska) (številska),}
\item{\verb|povprečno število zadetih prostih metov| (številska),}
\item{\verb|povprečno število vseh prostih metov| (številska),}
\item{\verb|odstotek zadetih prostih metov| (številska),}
\item{\verb|povprečno število skokov v napadu| (številska) ,}
\item{\verb|povprečno število skokov v obrambi| (številska),}
\item{\verb|povprečno število vseh skokov| (številska spremenljivka),}
\item{\verb|povprečno število asistenc| (številska),}
\item{ \verb|povprečno število napak| (številska),}
\item{\verb|povprečno število ukradenih žog| (številska),}
\item{\verb|povprečno število uspešnih blokad| (številska),}
\item{\verb|povprečno število vseh poskusov blokad| (številska),}
\item{\verb|povprečno število napravljenih osebnih napak| (številska) ,}
\item{\verb|povprečno število osebnih napak, napravljenih s strani nasprotnika| (številska),}
\item{\verb|povprečno število košev| (številska spremenljivka),}
\item{\verb|povprečna razlika v koših na koncu tekme (za igralce pomeni: povprečna razlika v koših, ko je igralec v igri)| (številska),}
\item{\verb|status ekipe (dobra, slaba,..)| (urejenostna spremenljivka),}
\item{\verb|zvezna država, v kateri ekipa domuje| (imenska spremenljivka).}
\end{enumerate}

\smallskip

Tabela \textbf{zmage} (v CSV obliki) vsebuje 3 stolpce. Za vsako zvezno državo so podani naslednji podatki:

\begin{enumerate}
\item{\verb|zvezna država| (imenska spremenljivka),}
\item{\verb|domače zmage| (številska spremenljivka),}
\item{\verb|gostujoče zmage| (številska spremenljivka).}
\end{enumerate}

\smallskip
Tabela \textbf{usa4} (v CSV obliki) vsebuje 5 stolpcev. Za vsako zvezno državo so podani naslednji podatki:

\begin{enumerate}
\item{\verb|zvezna država| (imenska spremenljivka),}
\item{\verb|glavno mesto zvezne države| (imenska spremenljivka),}
\item{\verb|geo. širina glavnega mesta| (številska spremenljivka),}
\item{\verb|geo. dolžina glavnega mesta| (številska spremenljivka),}
\item{\verb|konferenca, v kateri je posamezna zvezna država| (imenska spremenljivka).}
\end{enumerate}

\smallskip
Tabela \textbf{place} (v CSV obliki) vsebuje 2 stolpca. Za vsako ekipo so podani podatki o izdatkih za plače igralcem v sezoni 13/14:

\begin{enumerate}
\item{\verb|ekipe| (imenska spremenljivka).}
\item{\verb|izdatki za plače igralcem| (številska spremenljivka).}
\end{enumerate}

\smallskip
\textbf{Tabela ANA1} (v CSV obliki) prikazuje statistične podatke za ekipo PHOENIX Suns, in sicer za obdobje petih let:

\begin{enumerate}
\item{\verb|leto| (številska spremenljivka),}
\item{\verb|zmage| (številska spremenljivka),}
\item{\verb|skoki| (številska spremenljivka),}
\item{\verb|asistence| (številska spremenljivka),}
\item{\verb|napake| (številska spremenljivka),}
\item{\verb|točke| (številska spremenljivka).}
\end{enumerate}

\smallskip
\textbf{Tabela ANA2} (v CSV obliki) vsebuje podatke:
\begin{enumerate}
\item{\verb|sezona-leto| (številska spremenljivka),}
\item{\verb|število tujcev v ligi| (številska spremenljivka),}
\end{enumerate}
\smallskip


\textbf{Tabela NBA2} (v CSV obliki) prikazuje statistične podatke za ekipe v NBAs:

\begin{enumerate}
\item{\verb|zmage| (številska spremenljivka),}
\item{\verb|procent zadetih metov za dve| (številska spremenljivka),}
\item{\verb|sprocent zadetih metov za tri| (številska spremenljivka),}
\item{\verb|procent zadetih prostih metov| (številska spremenljivka),}
\item{\verb|skoki| (številska spremenljivka),}
\item{\verb|asistence| (številska spremenljivka),}
\item{\verb|ukradene žoge| (številska spremenljivka),}
\item{\verb|uspešne blokade| (številska spremenljivka),}
\item{\verb|točke| (številska spremenljivka),}
\item{\verb|razlika v danih in prejetih koših| (številska spremenljivka),}
\end{enumerate}
\smallskip

\textbf{Na naslednjih dveh straneh sledijo grafi.}

\includepdf[pages={1-2}]{../slike/Grafi.pdf}



\section{Analiza in vizualizacija podatkov}

V tretji fazi sem uporabil že prej uvožene podatke, pa tudi nove podatke, in na podlagi teh izdelal zemljevid.
\smallskip

Program sem napisal v skripto \verb|vizualizacija.r| v mapi \verb|vizualizacija| in ga vključil v \verb|projekt.r|, da se pokliče z ukazom \verb|Source|.
\smallskip

\textbf{Spremembe v uvozu}
V uvozu sem imel različne tabele z obilo podatki, vendar ti podatki niso povsem vezani na neko državo oz. zvezno državo, zato sem uvozil še nekaj tabel, in sicer \verb|nbacities|, \verb|zmage|, \verb|usa3|. Tabela \verb|nbacities| prikazuje ameriške zvezne države in mesta, kjer se igra košarkarska liga NBA. Na spletu sem poiskal koordinate za željena mesta, in sicer na spletni strani \url{http://www.infoplease.com/ipa/A0001796.html} ter jih vključil v mojo tabelo.

\smallskip

Tabela \verb|zmage| prikazuje domače in gostujoče zmage za posamezne zvezne države. V primeru, da je bilo v eni zvezni državi več ekip, sem podatke seštel in deljiv s številom ekip. Tako sem dobil nekakšno povprečje.
\smallskip

Tabeli \verb|usa3|, kjer so podatki za vse zvezne države, sem dodal stolpec division, kjer piše, kateri diviziji pripada posamezna zvezna država (tam , kjer se ne igra liga NBA so vrednosti NA).
\smallskip

\textbf{Vizualizacija}
\newline
Uvozil sem zemljevid \verb|USA| s pripadajočimi zveznimi državami. Zaradi boljše preglednosti sem odstranil tiste zvezne države, ki se ne držijo celinskega dela ZDA (npr. Aljaska, Havaji..), in shranil zemljevid kot \verb|usastates|. Na podlagi uvoženih podatkov sem nato narisal zemljevid ZDA, na njem pa označil mesta (zmagovalna ekipa sezone je obarvana z zlato), kjer se igra liga NBA. Zvezne države pa sem razdelil po divizijah (Pacific, Atlantic,..) in jih prav tako prikazal na zemljevidu.
\smallskip


Drugi zemljevid je sestavljen iz dveh delov. 
Na prvem zemljevidu so zvezne države, kjer se igra NBA, obarvane na podlagi učinkovitosti ekip na domačih tekmah, in sier temnejša barva pomeni več zmag.
Na prvem zemljevidu so zvezne države, kjer se igra NBA, obarvane na podlagi učinkovitosti ekip na gostujočih tekmah, in sier temnejša barva pomeni več zmag.\\
Cilj je bil prikazati dva zemljevida in grafično potrditi, da so ekipe bolj uspešne, ko igrajo na domačem terenu.

Z ukazoma \verb|plot| in \verb|spplot| sem ustvaril naslednja zemljevida:

\makebox[\textwidth][c]{
\includegraphics{../slike/USAcities.pdf}
}

\includepdf[pages={1-2}]{../slike/USAstats.pdf}



\section{Napredna analiza podatkov}

Prišel sem na idejo, da bi lahko analiziral podatke o številu tujih igralcev (international players), ki igrajo v ligi NBA oz. njihovo naraščanje skozi daljše časovno obdobje. Ker tovrstnih podatkov nisem našel na spletu v obliki, ki bi si jo želel (torej sistematično urejeni podatki po letih), sem podatke poiskal s pomočjo različnih spletnih virov, pa še to le za sezone od 2005 do danes. Za cilj sem si postavil odgovor na vprašanje, kakšen je trend igranja tujcev v ligi NBA, torej ali narašča, pada, stagnira. Zanimalo me je tudi, o kakšni številki tujih košarkarjev v ligi NBA lahko govorimo v bližnji prihodnosti.
\smallskip

Iz danih podatkov sem sestavil dva možna modela, ki bi lahko pokazala, kakšen je trend spreminjanja števila tujcev v ligi. Eden je linearni, drugi pa kvadratni model. Zanimalo me je, kateri se boljše prilega mojim podatkom. 
To lahko preverim s točno metodo oz. izračunom, kateri model se boljše prilega ali pa s sklepanjem iz grafa.

\smallskip

Kar se tega dela problema tiče, bom izračunal odstopanja napovedanih vrednosti od dejanskih (te vrednosti so preračunane vsote kvadratov razdalj od napovedanih do dejanskih vrednosti). Model z manjšo vsoto kvadratov razdalj bo tisti, ki se mojemu tipu podatkov bolj prilega.

\smallskip

Linearni izračun vrne vrednost 226.02424, medtem ko izračun za kvadratni model vrne vrednost 71.10758. Iz tega sledi, da se kvadratni model bolje prilega mojim podatkom. Če bi bilo podatkov več, bi model pokazal natančnejši prikaz naraščanja tujcev. 
Enačbi krivulj za modela sta naslednji:
\begin{enumerate}
\item{\verb|linearni model| ply = 1,83 x (leto - 2000) + 67,01, kjer spremenljivka ply pomeni število tujcev v ligi.}
\item{\verb|kvadratni model| ply = 0.5417 x (leto - 2000) x (leto - 2000) - 8.4614 x (leto - 2000) + 111.4288, kjer spremenljivka ply pomeni število tujcev v ligi.}
\end{enumerate}

\smallskip

\makebox[\textwidth][c]{
\includegraphics{../slike/internationalplayers.pdf}
}

Nato sem s temi podatki želel napovedati, kako se bo gibalo število tujcev v prihajajočih letih (sezonah). S funkcijo predict sem napisal novo funkcijo, ki napove podatke (odvisna od izbire modela: linearni, kvadratni). V zadnji NBA sezoni je bil postavljen rekord števila neameričanov v ligi, in sicer 101. Zanimalo me je, kdaj bo igralo 120 tujcev v ligi. Poiskal sem presečišča linearne oz. kvadratne funkcije s funkcijo y = 120 in prišel do naslednjih ugotovitev: po kvadratnem modelu bi naj to bilo v sezoni 2016/17;
po linearnem modelu pa v sezoni 2028/29.

\smallskip

\makebox[\textwidth][c]{
\includegraphics{../slike/napoved.pdf}
}

\newpage
V 4. fazi projekta sem se prav tako odločil raziskati, katere izmed statističnih spremenljivk so med seboj bolj korelirane in katere manj. Podatke sem jemal iz tabele \verb|ANA1|, in sicer zanimali so me podatki o zmagah, skokih, asistencah, točkah in napakah za ekipo Phoenix Suns (za obdobje petih let).

\smallskip

\makebox[\textwidth][c]{
\includegraphics{../slike/analiza.pdf}
}
\newpage
Kot izhodiščno spremenljivko sem si izbral zmage in gledal kako se druge spremenljivke obnašajo ob spreminjanju števila zmag. Če smo pozorni na \verb|zmage|, \verb|točke| (strelska učinkovitost) in \verb|napake|(izgubljene žoge), lahko iz grafa opazimo, da so te spremenljivke med seboj najbolj povezane. Krivulji točk in napak imata največji vrednosti v letih 2010 in 2014, in tudi število zmag je v omenjenih letih največje. Prav tako pa je v letu 2013 izrazito veliko napak in slaba strelska uspešnost ekipe, temu primeren pa je močan padec števila zmag. V košarki je tudi pomemben podatek o skokih, in sicer krivulja skokov ima največjo vrednost ravno v letih, kjer je število zmag izrazito veliko. V vmesnih sezonah pa ta spremenljivka ni preveč zanesljiva, da bi si samo na podlagi nje upali trditi o številu zmag (morda bi lahko povečal nabor podatkov in tako izboljšal samo analiziranje).
\smallskip

Kot opombo bi dodal, da sem podatke o asistencah množil z 2, podatke o številu točk pa delil s 3. Tako sem vse krivulje lahko prikazal na enem grafu in tako lažje opazoval, v kakšnem odnosu so spremenljivke ena do druge.

\medskip

Prav tako me je zanimalo, kateri izmed igralcev je bil najbolj učinkovit kar se števila košev tiče, in seveda minut, prebitih na parketu. Že v drugi fazi sem ugotovil, da so najbolj točkovno uspešni igralci ekipe Phoenix Suns Dragič, Bledsoe, Green in Morris. Spaševal sem se, ali je veliko število točk na tekmo le posledica visoke minutaže ali pa dejansko gre za odlične strelce (boljše od ostalih soigralcev). Indeks uspešnosti sem izračunal tako, da sem pri vsakem igralcu povprečno število točk delil z povprečno število prebitih minut na igrišču. Na koncu sem prišel do spoznanja, da imajo prav ti štirje najboljši igralci tudi največje indekse. Naslednji graf pa prikazuje učinkovitost (točke/minute) igralcev ekipe Phoenix Suns.
\medskip

\makebox[\textwidth][c]{
\includegraphics{../slike/indeksphx.pdf}
}

Na nasldenjih dveh grafih sem prikazal delitev igralcev ekipe Phoenix Suns glede na minute prebite na parketu, povprečje danih točk na tekmo, razmerje med danimi in prejetimi koši ekipe, ko je igralec na parketu ter drugimi statističnimi spremenljivkami iz tabele \verb|NBA2|. V skupini 1 z rdečo barvo so igralci, ki ne pripomorejo ekipi do velikih uspehov, bodisi zaradi slabe minutaže oz. slabe strelske učinkovitosti, morda tudi slabe igre v obrambi in drugih razlogov. V drugi skupini z modro barvo so igralci, ki največ pripomorejo k uspehu ekipe. V tretji skupini z zeleno barvo pa so igralci srednjega razreda, vendar vsekakor pomemben del ekipe. V pozitivnem smislu najbolj izstopata igralca v modri barvi, ki smo ju že prej potrdili za odlična, to sta Dragič in Bledsoe.

\makebox[\textwidth][c]{
\includegraphics{../slike/dendogram1.pdf}
}

\newpage
V naslednjem delu te zadnje faze sem se lotil raziskati še eno vprašanje. Narisal sem dendogram in ugotavljal, kako se v skupine razvrstijo NBA ekipe. Najprej sem skonstruiral tabelo \verb|NBA2|,kjer so podatki za vse NBA ekipe,
in kjer v stolpcih nastopajo le spremenljivke, kjer višja vrednost pomeni več oz. boljše (ni napak oz. izgubljenih žog, kjer več pomeni slabše).
\medskip

Nato sem s funkcijo \verb|scale| normaliziral podatke in narisal dendrogram.
Klube sem s funkcijo \verb|rect.hclust| razdelil v 3 skupine. Prav tako sem dendrogramu dodal legendo.
\smallskip

\makebox[\textwidth][c]{
\includegraphics{../slike/hierarhija1.pdf}
}

\newpage
Skupina 1 v rdeči barvi predstavlja najboljše klube v ligi, skupina 2 v modri pa najslabše, v skupini 3 v zeleni barvi so srednje dobri klubi.
Nato sem narisal še en dendrogram, ki nazorno prikazuje, katere ekipe so porabile največ in katere najmanj za plače. Med klubi, ki so porabili največ za plače igralcev so Brooklyn Nets, New York Knicks, Miami Heat, LA Lakers. V dendrogramu po učinkovitosti ekipe glede na več spremenljivk so te ekipe postavljene v 1. skupino (najboljše). Ekipe z manjšimi izdatki za plače in posledično z manj zvezdniki v ekipi so Philadelphia 76ers, Milwaukee Bucks, Utah Jazz, Sacramento Kings, Charlotte Bobcats, Detroit Pistons. Te ekipe so v dendrogramu glede na uspešnost uvrščene v 2. skupino (najslabše). Seveda prihaja tudi do izjem kot npr. ekipe San Antonio Spurs, Phoenix Suns in še kakšna, vendar je v grobem moč potrditi tezo, da so klubi z več denarja, dražjimi igralci konkurenčnejši od tistih z manj znanimi imeni.

\medskip

\section{Zaključek}
Za konec lahko povem, da sem z delom na projektu podrobneje spoznal, kateri izmed statističnih podatkov so bolj oz. manj pomembni v svetu košarke, pa tudi, da teren same tekme (domači, gosti) igra še kako pomembno vlogo v svetu športa. Prav tako sem narisal zemljevid ZDA in nazorno pokazal, kje se igra liga NBA, katera območja so morda bolj zapostavljena, kar se tiče profesionalne košarke, pa tudi v katero divizijo je razvrščena posamezna ekipa. Prav tako me je zanimalo, koliko neameričanov igra v ligi. Prišel sem do podatkov, da se število tujcev počasi, a vztrajno povečuje. V sezoni 14/15 je bil postavljen rekord, in sicer 101 igralec tujega porekla (neameričan). S funkcijo predict sem skušal z linearnim in kvadratnim modelom napovedati, kdaj bo v ligi igralo 120 tujcev in prišel do kar velikih odstopanj. Kljub temu, da bi naj bila kvadratna metoda natančnejša, se mi zdi, da je večja verjetnost, da bo omenjeno število tujcev v ligi igralo leta 2028, in ne 2017 kot pravi kvadratna metoda. Vzrok za tako velika odstopanja so v tem, da nisem imel velikega nabora podatkov na spletu (našem sem jih le od leta 2005 do danes). Na koncu sem še skušal raziskati, ali izdatki, ki jih klubi namenijo za plače igralcev, vplivajo na športne rezultate. Prišel sem do ugotovitve, da je v grobem to res, saj višje plače privabijo boljše igralce oz. zvezdnike z več kvalitetami, posledično pa so tudi rezultati boljši. Glavni cilj projekta je bil spoznati in delati z orodjem R Studio ter podrobno analizirati poljubne zbrane podatke. Mislim, da je projekt uspel in da sem pridobil izkušnje in znanja za delo na področju statistične analize podatkov.

\newpage
\begin{thebibliography}{9}
\bibitem{1}
  \url{http://sportslistoftheday.com/2013/12/15/nba-2013-14-team-payrolls-and-salary-cap-hits/}\\
  {Accessed: 01-03-2015}
\bibitem{2}
  \url{http://en.wikipedia.org/wiki/List_of_Olympic_Games_host_cities}\\
  {Accessed: 01-03-2015}
\bibitem{3}
  \url{http://stats.nba.com/league/team/#!/?Season=2013-14}\\
  {Accessed: 01-03-2015}
\bibitem{4}
  \url{http://stats.nba.com/league/team/#!/?Season=2013-14&Location=Road}\\
  {Accessed: 01-03-2015}
\bibitem{5}
  \url{http://stats.nba.com/team/#!/1610612756/players/?Season=2013-14}\\
  {Accessed: 01-03-2015}
\bibitem{6}
  \url{http://www.infoplease.com/ipa/A0001796.html}\\
  {Accessed: 01-03-2015}
\bibitem{7}
  \url{http://gadm.org/download}\\
  {Accessed: 01-03-2015}
\bibitem{8}
  \url{http://en.wikipedia.org/wiki/List_of_foreign_NBA_players}\\
  {Accessed: 01-03-2015}
\bibitem{9}
  \url{http://en.wikipedia.org/wiki/National_Basketball_Association}\\
  {Accessed: 01-03-2015}
\end{thebibliography}

\end{document}
